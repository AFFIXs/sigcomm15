\subsection{Naming}
\label{subsec-implementation-discussion}

\cappos{Does this belong earlier?   It ties in well with the deployment
subsection...}

An important aspect of democratic deployability is the ability to support 
communication with legacy (non-Affix) clients and servers.   In other 
words, Affix itself should be democratically deployable.   %To support
%this, there are several important challenges.
%\cappos{We likely need to discuss naming before this...}
However, on the Internet, hosts are commonly identified by 
a DNS name.   The DNS name is then mapped to an IP address.
However neither of these serve as a unique identifier
for a communication endpoint, especially in the presence of NAT boxes.
If we ignore non-Affix applications for the moment, it would be possible
to simply have an Affix endpoint use a Affix ID
as a name in a global registry similar to DNS.   (For scalability reasons,
such a registry can be implemented as a distributed hash table.)
The client could look
up the ID and use the resulting information (such as the serialized 
Affix stack) to understand how to contact the server.   However, this only
will work when both the server and client support Affix.
\cappos{I need your help here.   This text is rough to say the least.}

Applications must work correctly with either a client or a server does not 
support Affix.   If a server does not support Affix but the client does support
Affix, then it is trivial for the client to simply contact the server
without Affix.
However, an unmodified legacy client 
must be able to communicate with an Affix server.   If a legacy client knows
the server's IP or hostname instead of its Affix ID, the client can connect
to the server.
%the client does not know how to connect to a server given an Affix ID.

However, in some cases a server is not directly by IP address (eg when
behind a NAT).
For legacy clients that use a hostname, it is possible to still support
Affix by overloading the existing name.  To do this, a service called 
Zenodotus maps every
Affix ID to a valid DNS name.   A client using a Zenodotus name
(ending in {\tt zenodotus.[anonymized].org}) and it is resolved to an
IP address where the client may contact the server. 
%For example, in Figure~XXX(a), a webserver wants to support compression
%for Affix clients while also supporting legacy clients.   Legacy
%clients will look up the webserver's DNS name and contact it on 1.2.3.4, port
%80.   However, Affix clients will lookup the Affix ID and find that the
%server supports compression on port 8000.   An Affix enabled client can take
%advantage of this support.
%
%Note that the Affix ID for a server need not refer to the host's IP address.
For example, in Figure~XXX, a server behind a NAT is contactable by 
providing the
relay's IP address to legacy clients.   In this example, Affix clients are 
able to connect to the server.   Additionally, clients that support
Affix will benefit from compression and encryption while communicating
over the relay.

%Given a portion of the address space to map names into (such as exists in 
%IPv6), Affix could also support legacy clients that use IP addresses
%instead of hostnames.
%\cappos{Is this too tangental?}



%A legacy client can use the DNS name
%to uniquely refer to the server.  Zenodotus will resolve this to
%an IP address that a legacy client can use to communicate with the server
%(Figure~XXX).   This 
%Affix components that are transparent to the client.
%XXX
%
%However, a server may have different services running on different ports.
%If the same hostname is used, then the same public relay must be used
%for all ports on the server.   This may be less than ideal because different
%applications / flows may want to relay through different systems.   To
%mitigate this, the XXX




\cappos{I'm intentionally dodging ports / protocols because I think it muddies
the water too much.}
