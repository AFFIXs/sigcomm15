\section{Related Work}
\label{sec-related}
%\cappos{Perhaps move this entire section to the end.   I don't know what Affix
%really is yet and we don't need to build off of these.}

This work is heavily influenced by previous work 
with a similar modular structure or with similar ideas.
However, having explained in detail the specific goals of this 
work in Section~\ref{sec-goals}, 
previous approaches fall short of fulfilling our goals.

\subsection{Componentized networking frameworks}

Affix is similar in structure 
to other componentized networking frameworks such as 
x-kernel~\cite{hutchinson1991x,hutchinson23rpc,peterson1990x},
\textsc{Tesla}~\cite{salz2002tesla, salz2003tesla},
SILO~\cite{dutta2007silo, baldine2007unified},
Click~\cite{Fu_TOCS_2002},
and GNU Radio~\cite{blossom2004gnu}.
However, while these frameworks make
it easier to develop new networking services, 
they do not encourage their deployment,
because none satisfy the requirements for 
democratic deployability.

An early contribution, 
x-kernel~\cite{hutchinson1991x,hutchinson23rpc,peterson1990x},
is an operating system with a redesigned protocol stack 
to better support composition of networking services.
This approach, while highly efficient, does not support 
democratic deployability because it (by design) requires operating system 
support. Furthermore, x-kernel does not allow services to be 
added and removed dynamically when a communication is initiated.
  
Like Affix, \textsc{Tesla}~\cite{salz2002tesla, salz2003tesla} 
is an interposition layer that is situated 
between applications and the network 
stack on an operating system.
While it tries to provide a mostly uniform 
interface, \textsc{Tesla} is willing to break interface compatibility 
in some scenarios. For example, it explicitly breaks 
interface semantics in its mobility implementation. This 
in turn implies that certain permutations of \textsc{Tesla} layers 
involving the mobility layer would not be correct. 
\textsc{Tesla} lacks the ability to manipulate the layers in a 
flowgraph during runtime, so services must be coordinated
a priori.

Other frameworks including 
SILO~\cite{dutta2007silo, baldine2007unified},
Click~\cite{Fu_TOCS_2002},
and GNU Radio~\cite{blossom2004gnu},
have components that present varied interfaces.
As a result, permutability is not satisfied and components 
may be combined only in very particular ways.
While some dynamic addition and removal of services 
is supported -- Click, for example, 
allows a new router configuration to be installed during 
runtime -- individual components cannot be added, 
removed, or reordered arbitrarily. 

We have also seen componentized frameworks
for offering services in a specific domain, 
such as FRESCO~\cite{fresco_2013} for security services,
BlockMon~\cite{blockmon_2013} for monitoring services, 
and Pyretic~\cite{monsanto_composing_2013} for composing 
routing rules.
Like the more general componentized networking 
frameworks, these do not meet the requirements
for democratic deployability.

\subsection{Supporting Internet evolution}

Other prior work has addressed the goal of allowing 
the Internet and Internet services to evolve
more quickly, and without requiring consensus.

The properties of interface compatibility
and permutability described in Section~\ref{sec-goals}, 
and their relationship to the evolution of new services,
have been discussed in the context of other domains, 
such as composable simulation systems~\cite{petty_composability_2003}
and semantic web services~\cite{medjahed_multilevel_2005}.
Affix is the first work to apply these concepts
in the context of a componentized framework 
for networking services that can be installed on 
end-user devices, and to specifically address the problem 
of democratic deployability.

Similar goals have also been addressed in 
ANTS~\cite{ants,Wetherall_SOSP_1999},
a framework for programming active networks so that
new protocols can be deployed dynamically without 
a priori coordination. However, ANTS itself 
requires mass deployment by network service 
providers operating intermediate routers before 
end-users can benefit from these protocols.
In contrast -- and in line with the principles 
of democratic deployment --  Affix services can 
be deployed directly by end users on their own devices, 
on legacy networks with legacy applications,
without the involvement of service providers.

We have also seen proposed future Internet architectures
such as XIA~\cite{xia_2012} and FII~\cite{koponen_architecting_2011}
that, like Affix, aim to 
support the incremental deployment of new functionality, 
with a similar approach to many key issues.
However, unlike Affix, 
these are not compatible with the current
Internet, but are meant to guide the design 
of future networks.


\iffalse
The semantic problems we describe are not isolated to
situations where network functionality is added. It
is not uncommon for different operating systems to have behavior that
varies~\cite{IEEE:2001:ISRa}.
Semantic differences have been a major source of confusion 
both in academia~\cite{schmidt1995object, demmer2007towards, 
salz2002tesla} and in industry~\cite{BSDLinuxDiff, 
WinsockPort, NonBlockingDiff, BSDCompat, SOLINGERSemantics, CapposPythonBug1,
REUSEADDRSemantics}.

Our work is the first to validate semantic consistency when adding 
network functionality.   As a result, other work makes no guarantees about
the semantic validity of libraries or the correctness of their composition.



However, like any systems work, a great deal of our inspiration came from 
existing designs.   Our vision of a light-weight manner for adding network 
functionality has been realized in many previous incarnations. We categorize previous works as following:

\subsection{Separating Functionality}

Many pieces of prior work discussed dividing network functionality into 
components that can be added on demand.   
SILO~\cite{dutta2007silo, baldine2007unified} proposes a system that adds 
network functionality, but varies the interface for the different services. As 
a result, arbitrary interfaces are not stackable,  leaving a 
problem similar to that of conventional libraries.   
  

TESLA~\cite{salz2002tesla, salz2003tesla} works to provide a mostly uniform 
interface.   However, it is clear that semantics are not maintained when 
reading through their experiences.   For example, TESLA explicitly breaks 
the interface semantics to handle cases like mobility.
This implies that the TESLA layers cannot be stacked to create new 
functionality, one of the important goals of our work.  

%Demmer, et. al.~\cite{demmer2007towards} propose a message-based abstractionistency, but the abstraction is different from TCP and UDP, and therefore cannot be applied to legacy applications and networks. 

The x-kernel work~\cite{hutchinson1991x,hutchinson23rpc,peterson1990x} uses 
an abstraction that deals with efficiently passing network
data between different protocol objects on the local system.   This provides 
a high degree of network control along a very low level API, but requires
extensive operating system support.  x-kernel also fixes the network layers 
at boot time; therefore new functionalities cannot be added {\em on-the-fly}, 
and as the result the system cannot adapt to changing network and environment 
conditions.

\subsection{Middleware}

Our shim approach is inspired by several research efforts that use a 
intermediate layer between the application and the network stack, to 
achieve different goals. Our own work, ViFi~\cite{vifi}, uses a special 
driver to capture application packets to modify them. ViFi adds a header 
to each packet to provide additional network information to the remote 
server. 

The DTN2 Bundle Protocol~\cite{dtn_bundle} is another shim-like architecture 
proposed for enabling routing in sparsely connected networks. The bundle 
protocol resides between the application and the transport layer, and 
receives all application packets. The protocol aggregates packets into 
bundles and may route copies of a bundle to multiple nodes, to ensure 
that the packet get delivered in the sparse environment. 

%Moving beyond the networking context, a shim layer-like approach has 
%been used in operating systems and file systems. The Apple Filing 
%Protocol(AFP)~\cite{afp} intercepts file system commands and forwards 
%the system call based on whether the file resides locally or on a 
%remote server. 

Our goals for using a shim layer is different. We use the 
shim layer to develop an infrastructure that allows network engineers 
and programmers to easily add network functionality and build 
self-adapting applications. 

\subsection{Network functionality} 

Several researchers have worked on the problem of adding new 
functionality to networks. 

Active networks~\cite{tennenhouse1997survey} research introduced an 
architecture where network elements such as routers and switches make 
decisions for each packet according to some programming logic; in effect, 
adding network functionalities as packets are routed.  This requires changes
in the middle of the network, on devices with relatively limited computation
and tight timing constraints.  While shims could take advantages of this
functionality, shims can also operate on end hosts.   This makes 
deployment and adoption much easier. 

Similar to active networks, overlay networks~\cite{stoica2002internet} 
is another architecture that enables adding new network functionality.  
Overlay networks, as the name suggests, overlays an alternate network 
over the Internet to add additional network functionality such as 
security~\cite{ron}, QoS~\cite{overqos}, and reliable 
multicast~\cite{overcast}. Our shim architecture can easily encapsulate the
functionality rpovided by overlay networks and mask it from the application.
Shims also can add functionality to the end hosts.

Other solutions that add network functionality either assume that 
functionality that is added is transparent to the remote 
server~\cite{park2006connection,pucha2007exploiting, pucha2008adaptive, 
tolia2006architecture} or that the remote server already have the correct 
protocols in place~\cite{salz2002tesla, salz2003tesla, hutchinson1991x,
hutchinson23rpc,peterson1990x}. These assumptions restrict the kinds of 
network functionalities that can be added, severely affecting 
extensibility, one of our design goals.   

\subsection{Self-adapting protocols}

One of the primary goals of our proposed work is to design an 
infrastructure that will let applications easily adapt to changing 
network environments. 

There have been similar proposals to design self-adapting protocols, 
especially in the wireless network space. Our own work, R3~\cite{r3}, 
is a routing protocol that self adapts as network connectivity varies 
from being well-connected to being sparsely connected. The R3 protocol 
monitors the connectivity of the network to adapt the routing protocol. 
For example, R3 sends multiple copies of packets in sparsely connected 
networks, in the hope that at least one copy will reach the destination, 
but tunes down the number of replicas as the network becomes more connected.

Other related systems adapt link-layer rates according to 
mobility~\cite{ravindranath_nsdi11}, adapt the physical layer channel 
width according to the power of the wireless device~\cite{chandra_sigcomm08}, 
adapt routing in wireless mesh networks~\cite{rozner_tmc09}, etc. All 
of these systems focus on one aspect of diversity and one system variable, 
and adapt the system variable for the diverse condition. In contrast, we 
propose an infrastructure that will not only help combat current diversity 
challenges, but thats also extensible and can combat diversity challenges 
that may arise in the future. 

\fi

