\subsection{Coordination Service}
\label{subsec-services-coordination}
\cappos{Make this more formal.   Leverage the formalisms to define
a calculus for communication in this domain...}

Most Affix services operate on network traffic to improve performance, 
but unlike standard library or other implementations 
of these services, they can be deployed in a democratic fashion. 
However, they do not themselves contribute to democratic deployment. 
In contrast, the function of the \emph{Coordination} service is to 
enable democratic deployment. 

The Coordination service identifies the Affix components used 
by an opposite side of a communication flow, then loads a matching 
set of Affix components on the local host.
This enables communication through any composition
of services without prior arrangement.
The Coordination service described in this section is 
the exemplar service included in the Affix implementation and 
deployed on the Seattle testbed; 
it is also possible to implement alternative coordination services
with equivalent functionality but different properties.


\iffalse

While some Affix components can be used by either side of a 
connection transparently, many components will 
require some level of client-server communication in order 
to work properly. The client and server
needs to balance the Affix stack against each other in order 
to ensure proper communication. Take
for example that the server is using both the CompressionAffix 
and the EncryptionAffix in order
to compress and encrypt it's communication. If a client wants 
to connect and communicate with the
server, than they must also use the proper CompressionAffix 
and EncryptionAffix in order to be able
to decrypt and uncompress the data before returning the data 
to the application layer. In order for
the client to find out that the server is using these two 
components, it needs to be able to communicate
and coordinate with each other to ensure both sides are 
using the same Affix stack.

The practicality of different approachs taken to coordinate 
Affix stack information between clients and servers 
will depend on several factors. Consider the following 
two approaches.  

Approach One: A special purpose Affix component is put in place at the bottom of the network stack. 
Whenever a packet is received this componen willt facillitate communications to ensure the correct 
Affix stack is put into place on both sides of the connection.

Approach Two: A sever decides what Affixes will be used, and advertises the result under an applicaiton 
specific identifier, and port number in a secure look up service.  When a client program wants to connect 
to a server a look up is performed to find out what Affixes should be put into place. 

The first approach had the bennifit of not requireing an outside service, or the overhead added 
by advertisements and look ups.  As long as connection establishement is not an issue coordination 
of Affix stacks in this manner is perferable.  The second approach is, however, the only viable one 
in many situations. Consider for instance that no direct connection can be established to a particular server.  
The server can impose a NatPunchAffix that accepts traffic through a publicly reachable forwarder, 
but the client must some how learn what Affix stack to use in order to connect.  
\fi


The Coordination service implemented in Affix is always placed at the root
of an Affix tree. It redefines \texttt{listen()} to call
\texttt{get\_advertisement\_string()}. This returns a serialized description 
(called an \emph{Affix string})
of all non-transparent services underneath itself in the Affix tree.
The address at which it would like to be reached and the Affix string
with which it can communicate are sent as a key-value pair to 
a public \emph{advertise service}. Finally, the service invokes \texttt{listen()}
in the Affix component beneath itself.

When the opposite host calls \texttt{connect()}, the Coordination service 
queries the advertise service to find the Affix string stored for the 
given destination address. It then constructs a matching Affix tree beneath itself.

\albert{Fraidy noted: ``What is interesting? notable? about this coordination service?''}

\cappos{I was thinking something similar.   I was also thinking another
way to spin this is as protocol / service negotiation...}


\iffalse
Thus we implemented the second approach by leveraging an advertise service in order to
build the CoordinationAffix component which allows the client to do a lookup and build the
Affix stack the server is using, even if the server is directly unreachable. The advertise
service we use consists of multiple servers that are available on public domains. The service
has a redundancy system in order to ensure the service is always available.

The CoordinationAffix on the server side usually lies at the top of the stack and has a view
of all the Affixes that are in the stack. The CoordinationAffix serializes the stack on the
server side and advertises the serialized string on the advertise service. The CoordinationAffix
on the client side does a lookup on the advertise service to retrieve the serialized form of the
Affix stack. The coordination component than deserializes the Affix stack string and builds
the new Affix stack on the client side before making any network calls. Note that the 
CoordinationAffix only serializes the non-transparent Affixes that need to exist on both
sides of communication in order for the two nodes to communicate properly. Examples of such
components are CompressionAffix, EncryptionAffix and NatPunchAffix. The CoordinationAffix
however does not advertise transparent Affixes, such as LoggingAffix, StatAffix and RateLimitAffix.
\fi
